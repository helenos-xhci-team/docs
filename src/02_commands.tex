\section{Commands}

The xHC offers an independent command interface. During operation, the xHC
driver uses this interface to manipulate device slots, devices and endpoints by
executing various commands provided by a unified the command subsystem. This
section provides details on the structure and implementation of this subsystem.


\subsection{Execution Workflow}

The xHC command interface consists of a TRB command ring and a command doorbell
register. Commands are executed by placing various command TRBs onto this ring,
forming a \textit{Command Descriptor}.

After placing the respective TRBs and writing into the xHC command doorbell
register, the descriptors on the command ring are sequentially processed by the
xHC, resulting in either failure or successful completion. The result of every
command descriptor is reported back to the xHC driver in the form of a
\textit{Command Completion Event} placed onto the primary event TRB ring.

After writing into the xHC command doorbell register and before receiving the
respective command completion event, the xHC driver can attempt to abort the
issued command. Such action might be of use for instance if the command
completion event does not arrive within a set time period.

Note that this section intentionally omits hardware technical details, which are
not instrumental to understanding the command subsystem. For further hardware
documentation of the xHC command interface, refer to Section 4.6 of the xHCI
specification.


\subsection{Structure}

The xHC driver command subsystem instance is represented by the
\struct{xhci_cmd_ring_t} structure which exists throughout the entire duration
of the xHC driver's lifecycle. The purpose of this structure is to maintain and
manage the command TRB ring, and to keep track of enqueued command descriptors.

Individual command descriptors are represented by the \struct{xhci_cmd_t}
structure. In it are stored the high-level parameters of the command as well as
the command TRB, which is placed onto the command ring when the command is
executed. While the high-level command parameters are kept directly in this
structure, the hardware-related internals are kept in a substructure, which is
commonly referred to as \textit{command header}. The purpose of this separation
is to stress that the header contents are to be exclusively accessible to the
command subsystem only, while the rest of the structure remains accessible to
the entire xHC driver.

Besides data structures, the command subsystem offers a centralized command
completion event handler function -- \fnc{xhci_handle_command_completion()} --
which is called by the event subsystem in case a \textit{Command Completion
Event} is encountered.

The last major component of the command subsystem are functions used to generate
and schedule commands on the xHC. These functions produce valid instances of the
\struct{xhci_cmd_t} and place their respective TRBs onto the command ring
managed by the \struct{xhci_cmd_ring_t} structure, requesting either blocking or
non-blocking semantics for waiting on their completion. These functions are
described detail in the next section.


\subsection{Command Lifecycle}

\subsubsection{Issuing Commands}

By design, the internal logic of the command subsystem is kept opaque with
respect to other components of the driver. A notable example of this is the
mechanism for command scheduling.

Commands are usually issued by the HC component of the xHC driver. At the time
of issuing, the information required can be broken down into three groups:
~
\begin{description}
	\item[Command Type]
		One of the 15 commands supported by the xHC command interface.
	\item[Command-specific Parameters]
		The number, type and semantics of these parameters depends on the
		specified command type.
	\item[Completion Semantics]
		This is the desired behavior of command execution.
		
		In the \textit{blocking mode}, the call to issue the command will block
		the calling fibril until the completion event is received.
		
		On the other hand, in the \textit{non-blocking mode}, the fibril will
		only be blocked until the command is issued -- the command subsystem
		will take the ownership of the command and deallocate it when the
		completion event arrives. This mode is meant for commands, which do not
		require completion event handling, and involves more complicated memory
		management since the command subsystem is responsible for freeing the
		command after completion occurs.
\end{description}

Depending on the desired completion semantics of the issued command, the HC
component calls either the \fnc{xhci_cmd_sync()} or \fnc{xhci_cmd_async()}
function and pass it a configured instance of the \struct{xhci_cmd_t} structure.
Upon such call, the command subsystem will execute an internal command handler
function, which copies the high-level command-specific parameters configured by
the issuer, and use their values to construct a command descriptor consisting of
TRBs.

At this point, the \struct{xhci_cmd_ring_t} structure is modified and the new
command descriptor is placed onto the TRB ring. The corresponding instance of
\struct{xhci_cmd_t} is added to the active command list and depending on the
completion semantics, the issuing fibril is either suspended or continues
execution regardless of the completion event.


\subsubsection{Handling Completion}

When a command is completed, a \textit{Command Completion Event} is generated by
the xHC. This event is detected by the event subsystem, which in such case
triggers the \fnc{xhci_handle_command_completion()} function.

This function extracts the address of the completed command descriptor and uses
it to find a matching instance of the \struct{xhci_cmd_t} structure in the
active command list.

Depending on the desired completion semantics, the command subsystem either
wakes the sleeping issuer fibril, or discards the non-blocking command from the
memory.


\subsubsection{Aborting Commands}

% TODO @aearsis: Explain the arcane magic behind aborting commands.

