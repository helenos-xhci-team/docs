\chapter{USB Subsystem Modifications}
\label{usb-refactoring}

% TODO

\section{Explicit Device Removal}
One of the project goals is to alter the USB subsystem to allow support for
explicit device removal. Such feature can be found in most modern operating
systems and is often used to ensure that devices are left in a consistent state
after a physical port detachment occurs.

The explicit device removal feature usually provides a frontend interface in
the operating system, through which users can observe currently connected
devices and, if needed, issue a signal to the operating system that their
physical detachment is imminent. Following that, the system is expected to
promptly terminate all ongoing communications with the device and signal the
user back. After receiving the confirmation, user can then safely unplug the
device from the system bus without any risk of interrupting communications,
which could otherwise result in undefined state of the device.


\subsection{Offline and Online Signals}
The HelenOS Device Driver Framework includes two user-initiated signals
relevant to the implementation of this feature.

\begin{description}
	\item[Offline Signal]
		This signal informs a driver attached to a DDF node that its managed
		device may be removed in the near future. The driver is expected to
		immediately cease all user operations on the device and unbind its
		child DDF functions, possibly sending a \textit{Device Remove} signal
		to all their attached drivers in the process.

	\item[Online Signal]
		This signal is a logical counterpart to the previous signal.
		It informs a driver attached to a DDF node that its managed device will
		not be removed in the near future. The driver is expected to expose all
		child DDF functions related to the device, possibly sending a
		\textit{Device Add} signal to all their matched drivers in the process.
\end{description}

These signals can be easily issued by the user from the system shell by means
of the \app{devctl} application. See Listing \ref{lst:devctl-offline-online}
for invocation example.

\begin{listing}[H]
	\begin{bdsh}
		# Prepare the unplug high speed device at address 2.
		devctl offline /hw/pci0/00:04.0/usb2-hs

		# We changed our mind. Bring the device back online.
		devctl online /hw/pci0/00:04.0/usb2-hs
	\end{bdsh}
	\caption[Example usage of \app{devctl} to issue offline and online
	signal.]{Example usage of the \app{devctl} application to issue offline and
	online signal to a USB high speed device at address 2. The host controller
	PCI address is \texttt{00:04.0}.}
	\label{lst:devctl-offline-online}
\end{listing}

It follows that these signals can be used for the implementation of the
explicit device removal at the level of USB host controller drivers.


\subsection{Subsystem Changes}
% TODO


\subsection{UHCI Support}
% TODO


\subsection{OHCI Support}
% TODO


\subsection{EHCI Support}
% TODO


\subsection{xHCI Support}
% TODO

