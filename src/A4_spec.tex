\chapter{Project Specification and Timeline}

This section contains the original project specification, as it was approved by
the project committee at Faculty of Mathematics and Physics, Charles University.
The actual project realization timeline is also included.


\section{Project Specification (in Czech)}

\subsection{Základní informace}

\begin{description}
	\item[Jméno projektu] HelenOS USB 3.x Stack
	\item[Zkratka] HelUSB3
	\item[Vedoucí] Martin Děcký \email{decky@d3s.mff.cuni.cz}
	\item[Konzultanti] --
	\item[Anotace]
		Rozšíření podpory sběrnice USB a USB zařízení v mikrojádrovém multiserverovém operačním systému HelenOS o revizi 3.0 (resp. 3.1), sjednocení této podpory s dříve implementovanou funkcionalitou pro USB 1.0, 1.1 a 2.0 a další vylepšení.
\end{description}

\subsection{Motivace}

Podpora sběrnice USB v revizi 1.0 a 1.1, která byla naimplementována v rámci Softwarového
projektu v roce 2011 (HelUSB), výrazně rozšířila praktickou použitelnost operačního systému
HelenOS v souvislosti s periferními zařízeními. Na tuto funkcionalitu bylo později navázáno
podporou USB 2.0. V současné době je poslední specifikovaná revize USB 3.0 (z hlediska
hardwarového transportu potom 3.1) a začínají se objevovat počítače, které implementují pouze tuto
revizi (tedy z pohledu operačního systému nejsou zpětně kompatibilní se staršími revizemi). Proto
je vhodné, aby byl operační systém HelenOS doplněn o nativní podporu pro USB 3.0 (resp. 3.1).
Tento projekt je pochopitelně také vhodnou příležitostí pro provedení sjednocení s předchozími
implementacemi a provedení dalších souvisejících vylepšení.

\subsection{Popis projektu}

Primárním předmětem projektu je rozšířit generický framework pro použití USB sběrnic a
implementaci ovladačů USB zařízení v systému HelenOS o podporu USB revize 3.0 (resp. 3.1), při
zachování plné podpory předchozích revizí 1.0, 1.1 a 2.0. Cílem je, aby bylo možné na běžně
dostupném hardwaru využít nejvyšší možnou revizi USB s přihlédnutím k možnostem daných
periferních zařízení. Součástí projektu je také implementace základního demonstrátoru – ovladače
konkrétního fyzického USB host controlleru a ověření funkcionality konkrétních koncových USB
zařízení.

\begin{itemize}
	\item
		Implementace ovladače pro USB host controller podle normy xHCI.
		\begin{itemize}
			\item Podpora přenosových režimů a rychlostí odpovídajících USB 1.0, 1.1, 2.0 a 3.0.
			\item
				Enumerace zařízení na sběrnici USB 3.0 a spouštění příslušných ovladačů koncových USB zařízení (s využitím existujícího Device Driver Frameworku), jsou-li k dispozici.
				\begin{itemize}
					\item Zachování kompatibility s dříve naimplementovanými ovladači host controllerů podle norem OHCI, UHCI a EHCI.
					\item Ideálně zachování možnosti implementovat ovladače koncových USB zařízení nezávisle na použité variantě host controlleru.
				\end{itemize}
			\item Demonstrátor: Ovladač pro xHCI host controller NEC Renesas uPD720200
		\end{itemize}

	\item
		Implementace explicitního mechanismu odpojování USB zařízení (očekávaného i neočekávaného).
		\begin{itemize}
			\item Podpora přerušení USB komunikace (USB communication abort) na hardwarové úrovni bez zablokování ovladače USB zařízení.
		\end{itemize}

	\item Implementace podpory isochronního režimu komunikace USB zařízení.
	\item Volitelná část zadání: Rozšíření USB frameworku o dosud nepodporované vlastnosti (např. správa napájení), o podporu specifických xHCI host controllerů (např. Intel Sunrise Point-H a/nebo contollery integrované na vývojových deskách či SoC čipech typu BeagleBoard, BeagleBone, Raspberry Pi) či jiná vylepšení (např. implementace nových ovladačů koncových USB zařízení jako je DisplayLink, dokončení implementace správy přenosového pásma a výkonu, odladění ovladačů controllerů na jiných platformách).

\end{itemize}

\subsection{Platforma, technologie}

\begin{itemize}
	\item
	Framework a ovladače budou odladěny v simulátoru QEMU a na běžném fyzickém PC
	(obojí pro platformy x86 a x86-64) vybaveném výše uvedeným USB controllerem a
	případně kombinací již dříve podporovaných USB controllerů s použitím již existujících
	ovladačů koncových USB zařízení.

	\item
	Framework a ovladače budou implementovány takovým způsobem, který nebude
	principielně bránit jejich budoucímu využití na jiných platformách než x86 a x86-64.

	\item
	Framework a ovladače budou implementovány takovým způsobem, aby zachovávaly
	celkovou softwarovou architekturu mikrojádrového multiserverového operačního systemu
	HelenOS a aby nedošlo k omezení již dříve naimplementované a odladěné funkcionality (tj.
	podpora OHCI, UHCI atd.).
\end{itemize}


\subsection{Odhad náročnosti}

Na základě zkušeností z dřívějšího Softwarového projektu HelUSB (implementace podpory OHCI,
UHCI) lze říci, že zadání je řešitelským týmem o 5 – 6 studentech zvládnutelné ve standardní době.
Projekt klade na řešitele především následující nároky, ze kterých přirozeně plyne přibližný
harmonogram prací:

\begin{itemize}
	\item Nastudování specifikace sběrnice USB, specifikace xHCI, specifikace controlleru NEC Renesas uPD720200, volitelně studium implementací v jiných operačních systémech. [1 měsíc]

	\item Nastudování softwarové architektury, relevantních mechanismů a existující implementace systému HelenOS (async framework, Device Driver Framework, podpora OHCI, UHCI, EHCI). [1 měsíc]

	\item Implementace podpory xHCI a controlleru NEC Renesas uPD720200, integrace s existující implementací, refactoring. [2 měsíce]

	\item Implementace explicitního mechanismu odpojování USB zařízení, přerušení USB komunikace a isochronního režimu komunikace. [1,5 měsíce]

	\item Implementace vhodné podmnožiny volitelných částí zadání. [2,5 měsíce]

	\item Důkladné odladění výsledné implementace, sepsání dokumentace. [1 měsíc]
\end{itemize}

Aktuální stav systému HelenOS poskytuje dostatečně stabilní základ pro úspěšnou realizaci
projektu, riziko nedokončení projektu je malé. Zadání projektu záměrně klade důraz na elegantní
integraci výsledného řešení s existujícím kódem včetně nutného refactoringu. Nelze pochopitelně
zcela vyloučit nutnost opravovat drobné chyby v existující implementaci, které mohou být
v průběhu práce na tomto projektu odhaleny, nemělo by se však jednat o zásadní překážky.

Protože časovou náročnost implementace povinných bodů zadání není možné předem zcela
spolehlivě odhadnout, počítá zadání také s volitelnými částmi, které by posloužily jednak pro
zlepšení celkové funkcionality USB v systému HelenOS a jednak umožnily dosáhnout za všech
okolností obvyklého objemu implementačních prací v rámci Softwarového projektu. V případě, že
povinné části zadání zaberou odhadovaný čas, bude možné implementovat vhodnou podmnožinu
volitelných částí. Ukáže-li se naopak, že odhad časové náročnosti povinných částí byl
podhodnocen, resp. nadhodnocen, potom bude možné omezit, resp. rozšířit, podíl implementace
volitelných částí.


\subsection{Vymezení projektu}

Projekt je zaměřen na následující oblasti:
%
\begin{itemize}
	\item softwarové inženýrství,
	\item vývoj software,
	\item systémové programování,
	\item spolehlivé systémy.
\end{itemize}


\section{Project Timeline}

This section contains the full timeline of project realization throughout the
years 2017 and 2018.

\begin{table}[h]
	\centering
	{\renewcommand{\arraystretch}{1.2}
	\newcommand{\hl}{\rowcolor[gray]{0.95}}
	\begin{tabular*}{0.95\textwidth}{rr@{\hspace{\tabcolsep}}l|l}
	\hline \hl \textbf{2017}
	    & \textbf{April}    & \textbf{18} & Project specification finalized. \\
	\hl & \textbf{May}      & \textbf{1}  & Project officially started. \\
	    & May               &             & Studying specifications, getting into the context. \\
	\hl & \textbf{June}     & \textbf{8}  & First commit in the project branch. \\
	    & June              &             & Skeleton of the xHCI driver structure. Supportive structures. \\
	    & July              &             & Command interface. \\
	    & August            &             & Root hub and event ring. First signs of dialog with the HC. \\
	    & October           &             & Transfers, debugging. The great usbhost refactoring. \\
	\hl & \textbf{October}  & \textbf{17} & USB mouse driver transmitting movement data over the xHCI stack. \\
	    & November          &             & Debugging, refactoring.  \\
	\hl & \textbf{November} & \textbf{12} & HelenOS 0.7.1 was released. Changed VCS from Bazaar to GIT. \\
	    & December          &             & Transfer Monitor, testing, benchmarking. \\
	\hline \textbf{2018}
	    & January           &             & Device removal, isochronous transfers, USB 3 extensions. \\
	    &                   &             & Testing, debugging, writing documentation. \\
	\hl & \textbf{January}  & \textbf{30} & Project considered delivered by the supervisor. \\
	    & February          &             & Merging into mainline, finishing touches. \\
	\hl & \textbf{February} & \textbf{??} & Project submitted for defense. \\
	\hline
	\end{tabular*}
	}
	\caption{Project timeline.}
	\label{tbl:timeline}
\end{table}

