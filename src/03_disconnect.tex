\section{Explicit Device Removal}

One of the project goals is to alter the USB subsystem to allow support for
explicit device removal. Such feature can be found in most modern operating
systems and is often used to ensure that devices are left in a consistent state
after a physical port detachment occurs.

The explicit device removal feature usually provides a frontend interface in
the operating system, through which users can observe currently connected
devices and, if needed, issue a signal to the operating system that their
physical detachment is imminent. Following that, the system is expected to
promptly terminate all ongoing communications with the device and signal the
user back. After receiving the confirmation, user can then safely unplug the
device from the system bus without any risk of interrupting communications,
which could otherwise result in undefined state of the device.


\subsection{Considerations}

In the USB protocol, communications between the host and the device take place
in the form of \textit{transfers}. Depending on its version, the host controller
may have various roles in the realization of these transfers. For that reason,
version-specific modifications are carried out separately in host controller
drivers, whereas common functionality is implemented in the bus module, which is
a part of \lib{libusbhost}.

The disconnection routine for explicit device removal is implemented as follows:
~
\begin{enumerate}
	\item The user signals the intention to disconnect a USB device.
	\item The respective device drivers are notified to end their business (e.
		g. flush buffers or close files), possibly scheduling a multitude of
		transfers to the device.
	\item The HC driver disables the capability to schedule new transfers to the
		device.
	\item The HC driver aborts all leftover active transfers to the device.
	\item The device configuration is dropped, leaving it in the
		\state{Addressed} state, in which it is considered safe to be
		physically removed from the bus.
\end{enumerate}

If the user requests that this routine is rolled back, the steps of the
disconnection routine are just executed in reverse order. The following can
therefore be labeled as a reconnection routine:
~
\begin{enumerate}
	\item The user signals the intention to resume communications with a USB
	device, on which the disconnection routine has been previously performed.
	\item The HC driver configures the device.
	\item The HC driver enables the capability to schedule new transfers to the
	device.
	\item The operating system is notified that the device is reachable and
	matches it to appropriate drivers, which initiate communications with it.
\end{enumerate}

The specialization of both routines is performed in the same way as other bus
module interactions. All HC drivers hand off their DDF and device callbacks to
the bus module, which then calls them back to perform low-level commands related
to specific devices, endpoints and transfers. This way, the high-level logic
contained by the bus module essentially follows the listed descriptions above
and the version-specific extensions are resolved in the respective HC driver
implementations.


\subsection{Offline and Online DDF Signal}

The HelenOS Device Driver Framework includes two user-initiated signals
relevant to the implementation of this feature.

\begin{description}
	\item[Offline Signal]
		This signal informs a driver attached to a DDF node that its managed
		device may be removed in the near future. The driver is expected to
		immediately cease all user operations on the device and unbind its
		child DDF functions, possibly sending a \textit{Device Remove} signal
		to all their attached drivers in the process.

	\item[Online Signal]
		This signal is a logical counterpart to the previous signal.
		It informs a driver attached to a DDF node that its managed device will
		not be removed in the near future. The driver is expected to expose all
		child DDF functions related to the device, possibly sending a
		\textit{Device Add} signal to all their matched drivers in the process.
\end{description}

These signals can be easily issued by the user from the system shell by means
of the \app{devctl} application. See Listing \ref{lst:devctl-offline-online}
for invocation example.

\begin{listing}
	\begin{bdsh}
		# Prepare the unplug high speed device at address 2.
		devctl offline /hw/pci0/00:04.0/usb2-hs

		# We changed our mind. Bring the device back online.
		devctl online /hw/pci0/00:04.0/usb2-hs
	\end{bdsh}
	\caption[Example usage of \app{devctl} to issue offline and online
	signal.]{Example usage of the \app{devctl} application to issue offline and
	online signal to a USB high speed device at address 2. The host controller
	PCI address is \texttt{00:04.0}.}
	\label{lst:devctl-offline-online}
\end{listing}

It follows that these signals can be used for the implementation of the
explicit device removal at the level of USB host controller drivers. For that
reason, \lib{libusbhost} has been extended to handle appropriate DDF callbacks
for functions corresponding to HC's child devices. Their handling is forwarded
to the bus module, which executes the disconnection or the reconnection routine
for the \textit{offline} and \textit{online} signal respectively. In addition,
the transfer scheduling mechanism of the bus module has been extended to permit
scheduling new transfers only to devices, which are in the online state.

The general scheme of stopping communication with a device breaks down to
unregistering all its registered endpoints. The biggest challenge the driver
faces is to abort all currently running transfers on an endpoint that is being
unregistered. The majority of transfers (Bulk, Control, Isochronous,
Interrupt-out) wouldn't pose a problem -- we could just wait the short while
until they are completed, either successfully or not. But then there are
Interrupt-in transfers, which, especially in case of gone device, may not
complete in a timely manner.

\subsection{Aborting Active Transfers}

It is not possible to ``abort a transfer'' in a generic way, mainly because of
synchronization issues. Before we explain how can a transfer be properly
aborted in various Host Controllers, let us describe the lifecycle of
a transfer batch, a structure representing a transfer in HC drivers.

Currently, USB stack in HelenOS only supports synchronous interface to interact
with pipes. The two methods are called \fnc{dev_read} and \fnc{dev_write}.
Driver calls these methods on pipes, and provides a buffer -- either filled
with data, or to be filled. Once the call crosses the IPC barrier, it is joined
to a call to \fnc{bus_device_send_batch}. This function finds the target
endpoint structure, and passes control to \fnc{endpoint_send_batch}.

There, an instance of \struct{usb_transfer_batch_t} structure is created and
filled with parameters of the transfer. It is then passed to the driver
implementation to be scheduled. The driver typically copies the data to
a buffer suitable for the device, prepares some supporting structures, and
finally, schedules the transfer to the hardware.

Since then, an interrupt may come and finish the transfer in a different
fibril. A transfer is finished by copying the data out from the hardware buffer
to the batch buffer, setting the error code and calling a completion callback.
This callbacks answers the original incoming IPC call, causing the \fnc{dev_read/write}
function to return. After that, the transfer batch is destroyed.

But that's not the only scenario that may happen. From the moment a transfer is
created, a pointer to it must not be forgotten, otherwise the caller would
never return. But on the other side, once the pointer to batch is stored
somewhere, the transfer might be aborted at any time. Furthermore, once the
transfer is scheduled to the hardware, the buffers must not be deallocated
until the driver is sure that the hardware won't use them anymore.

This synchronization problem might be resolved by locking the batch and
reference counting, but then different problems would arise (e.g. a transfer
could be finished after the endpoint was successfully unregistered, just
because we cannot know if there's any). So, we decided to take a different
approach.

As the interface is synchronous (and it doesn't make much sense to make it
asynchronous, unless under special conditions), and the endpoint is assumed to
be available for one driver only, there's no point in having more than one
transfer active at a time. So, we store the pointer to a batch inside the
endpoint structure, in the field \struct{active_batch}. This field shall not be
accessed, unless the endpoint guard is locked.

Speaking about the endpoint guard leads us to one of the strange design
decisions we made. Endpoints do not have their own guard, they inherit one
while being put into the online state. That is, when the endpoint is being
registered, the HC driver calls \fnc{endpoint_set_online} and passes a pointer
to a mutex which will be used to synchronize transfers on that endpoint. The
endpoint itself never locks this mutex, it only uses it to ensure correctness
(whether the mutex is locked in named functions) and to wait on a condition
variable.

Sooner or later in the scheduling of a transfer batch, there is a need to access
shared structures. Once there is, the HC driver locks the mutex, and in a single
critical section it calls the function \fnc{endpoint_activate_locked}. If this
function successfully returns, the endpoint is reserved for a given batch. By
calling the function, the batch ownership is given to the endpoint -- once the
critical section ends, the batch must be considered already finished.

In the meantime, there are two possible execution flows that are related. First
of all, there is the interrupt handler finishing transfers. Once it decides
a transfer is to be finished, it is supposed to have the mutex already locked
(to avoid racing with the scheduling fibril), and calls function
\fnc{endpoint_deactivate_locked}. This function allows another batch to use the
endpoint, and transfers the ownership of the batch to the caller.

Second, there can be a fibril trying to unregister the endpoint. To do so, it
locks the mutex, and calls \fnc{endpoint_set_offline}. This blocks access for
further transfers, and also wakes fibrils waiting inside
\fnc{endpoint_activate_locked} from the sleep, returning an error value. The
unregistering fibril then may decide to wait a while to finish already running
transfers, and then do HC-specific steps to remove the running transfer from the
hardware. Eventually, it shall call \fnc{endpoint_deactivate_locked}, which
takes the ownership of the batch, and gives the unregistering fibril a sole
right to finish the transfer with an error code.

This whole mechanism is completely opt-in, and the driver can avoid using it
(like xHCI do for stream-enabled endpoints). The HC-specific part of the
implementation is discussed in the next sections.

\subsection{UHCI, OHCI and EHCI Specifics}

All three HCI's that were supported prior to our project have similar
structure with regard to what is required to implement transfer aborting. Let us
first describe very briefly how these controllers handle transfers.

Generally, all three host controllers require the driver to create a system of
linked structures in memory (for UHCI and OHCI, restricted to the lower 4 GBs
of addressable space). The names and guts differ, the structures however
describe a linked chain of queues. Queues are then filled by transfer
descriptors, which describe USB transfers to be done. Once the transfer is
done, its descriptor is flagged, removed from the queue and if the descriptor
is marked, the host is interrupted. More specific information can be found
either in respective specifications, or in the documentation of the HelUSB
project.

At the time of receiving an interrupt, the host does not know which transfer
was finished, so it has to check all pending transfer descriptors for the
completion flag. In case all the transfer descriptors of a transfer are
completed, the driver may finish the transfer.

When aborting a transfer, the driver must make sure the controller is not using
any of the allocated buffers. It is allowed to modify the queues while the
controller is scanning them, the modifications must however follow an order in
which the consistency of the structure is guaranteed at any time. After it
does, it must notify the driver that the structure has changed, in order to
force the controller to clear its caches (EHCI only).

Because the interrupt handler is polling the transfers for completion, it is
enough for the driver to remove the transfer from the list of pending transfers
to be sure no other fibril will ever complete it. After that, we can finish it
ourselves with an error. Note that in this case it is unknown to driver whether
the transfer was completed or not -- but since the driver is unregistering an
endpoint, the device must already be in a state in which it expects removal.

The nature of handling finished transfers defines the weird semantics of the
inherited mutex. Let's consider a situation, in which two locks were involved:
a lock protecting the HC structures (lists, interrupt handling, \dots) called
$H$ and a lock protecting the endpoint $E$. When a transfer is scheduled, it
must first wait until the endpoint is available. To avoid a deadlock between
finishing the current transfer and waiting for it to finish, the mutex $E$ must
be taken first. Also, $E$ cannot be released before taking $H$, because after
$E$ is released the transfer can be aborted immediately. On the other hand, when
an interrupt comes, there is no way how to get a pointer to the endpoint without
taking $H$ first. Neither $H$ cannot be released before taking $E$, because
we cannot access the batch unless holding $E$, and even transferring a pointer
to an endpoint does not help -- in between the critical sections the current
transfer transfer can be aborted and a new one scheduled, resulting in
completing a wrong transfer. To avoid ABBA deadlock eventually, we just have to
avoid using two locks for transfer synchronization.

From the further perspective, these controllers do not have an internal state
for individual devices and endpoints, so the deconfiguration and its rollback
is an operation on software-state only. As such it is already done completely
by the \lib{libusbhost} library.

\subsection{xHCI Specifics}

Since xHCI is the latest HC interface implementation, a lot more is done by the
hardware for the HC driver in comparison with previous versions. The concept of
xHCI command ring leads to very elegant implementation of the required
functionality on the HC driver part.

For the purpose of aborting active transfers, the xHCI features an explicit
\textit{Stop Endpoint} command, which instructs the HC to abort all transfers
to a specific device endpoint. This command is issued by the HC driver for all
removed device endpoints, which are active at the moment of the request.
Furthermore, device configuration is dropped along with all remaining endpoints
by issuing a \textit{Configure Endpoint} command with the DC (deconfigure)
flag.

Reconnection is quite straightforward and requires only that the HC driver
issues a regular \textit{Configure Endpoint} command in order to transition the
device from the \state{Addressed} state back to the \state{Configured} state.


\subsection{Driver Support}

Since the existing USB drivers were quite incomplete, their implementation has
been extended to add support for explicit device removal. This mostly lead to
novel approaches to deallocation of device-related memory structures, which is
often performed in the \fnc{device_remove()} and \fnc{device_gone()} driver
callbacks.

For instance, a number of drivers required that an implementation of
\fnc{device_remove()} is created in the first place. Quite so often, the
existing implementation of \fnc{device_gone()} has provided a good starting
point, given that both functions have to deal with USB device's demise. The
fundamental difference was that while \fnc{device_gone()} merely dealt with
the fallout of unexpectedly unplugged device, \fnc{device_remove()} had the
opportunity to tie up all lose ends prior to the physical detachment of the
device.

This posed a problem, especially in HID and hub drivers, which heavily relied on
polling of interrupt endpoints. Since polling was a synchronous operation from
the device driver's point of view, a polling fibril has been created when the
driver started managing the device. When the device was unplugged, the polling
operation failed with an error, waking up and effectively terminating the fibril in
the process. Since no explicit joining mechanism was available, the
implementation of \fnc{device_gone()} merely spinned a limited number of times,
waiting for the polling fibril to die. This mechanism was however unsuitable for
\fnc{device_remove()}, since the fibril would not awaken due to an error caused
by the physical disconnect, which has not happened yet.

This lead to complete refactoring and extension of the USB device polling
mechanism, which is described in detail in Section \ref{polling-refactoring}.
In summation, the new version of the mechanism allows device drivers to join
their polling fibrils and consistently wake them up in order to avoid deadlocks.

To actually stop the polling in the \fnc{device_remove()} callback, the device
driver has to trigger a transfer abort inside the HC. There's no sensible way
of how to allow a driver abort its own transfer, which wouldn't introduce
potential memory leaks inside HC or synchronization problems. So we decided
that the driver will have to unregister the endpoint as the only way to abort
a currently running transfer (and also disable scheduling of new ones). The
device's endpoints (or, at this layer already called pipes) are completely
managed by the \lib{libusbdev} library. Previously, their lifetime was strictly
tied to the lifetime of the handled device. There are only two possible
ordering of these two operations:

\begin{enumerate}
	\item Close the pipes after the \fnc{device_remove()} callback returns.
		After that, the polling fibrils will end and destroy their structures.
		The library USB device structure will have to be reference-counted and
		the counter will be managed by the driver to account for polling
		fibrils.

	\item Close the pipes prior to calling \fnc{device_remove()}, effectively
		destroying the only reason why this callback exists, and make the
		expected removal equivalent to the unexpected one.
\end{enumerate}

\noindent It is pretty obvious that we had to choose a completely different approach. So
we came up with three more options:

\begin{enumerate}
\setcounter{enumi}{2}
	\item Introduce a new callback, e.g. \fnc{device_removed()}, to be called
		after the pipes are closed. The removal would be split into two phases:
		first, in \fnc{device_remove()} the loose ends are closed and the
		device is brought to a state expecting removal, then in
		\fnc{device_removed()}, the polling fibrils are joined and structures
		are destroyed.

	\item After calling \fnc{device_remove()} and closing the pipes, call
		\fnc{device_gone()}.

	\item Allow the driver to close individual pipes imperatively.
\end{enumerate}

At first, we decided to go with option 3). But it was yet another callback, we
didn't come up with a name that wasn't so similar and yet was clear, and
finally, the \fnc{device_removed()} callback implementations were very similar
to \fnc{device_gone()}.

The option 4) seems reasonable at the first look, but it would create
inconsistency between DDF drivers and USB drivers, which would be very
confusing for developers. \footnote{Actually, some team members think this
behavior shall be present in DDF too, but we didn't open a discussion with
HelenOS developers.}

In the end, the only option left was letting the driver close its own pipes. So
we introduced a new library call \fnc{usb_device_unmap_ep()}, which does
exactly that. The new library polling mechanism closes its pipe automatically,
but this functionality is usable in any driver for any endpoint.

The implementation of explicit device removal support for the rest of the
drivers was a trivial extension of their previous functionality.

