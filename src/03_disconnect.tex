\section{Explicit Device Removal}

One of the project goals is to alter the USB subsystem to allow support for
explicit device removal. Such feature can be found in most modern operating
systems and is often used to ensure that devices are left in a consistent state
after a physical port detachment occurs.

The explicit device removal feature usually provides a frontend interface in
the operating system, through which users can observe currently connected
devices and, if needed, issue a signal to the operating system that their
physical detachment is imminent. Following that, the system is expected to
promptly terminate all ongoing communications with the device and signal the
user back. After receiving the confirmation, user can then safely unplug the
device from the system bus without any risk of interrupting communications,
which could otherwise result in undefined state of the device.


\subsection{Considerations}

In the USB protocol, communications between the host and the device take place
in the form of \textit{transfers}. Depending on its version, the host controller
may have various roles in the realization of these transfers. For that reason,
version-specific modifications are carried out separately in host controller
drivers, whereas common functionality is implemented in the bus module, which is
a part of \lib{libusbhost}.

The disconnection routine for explicit device removal is implemented as follows:
~
\begin{enumerate}
	\item The user signals the intention to disconnect a USB device.
	\item The respective device drivers are notified to end their business (e.
		g. flush buffers or close files), possibly scheduling a multitude of
		transfers to the device.
	\item The HC driver disables the capability to schedule new transfers to the
		device.
	\item The HC driver aborts all leftover active transfers to the device.
	\item The device configuration is dropped, leaving it in the
		\textit{addressed} state, in which it is considered safe to be
		physically removed from the bus.
\end{enumerate}

If the user requests that this routine is rolled back, the steps of the
disconnection routine are just executed in reverse order. The following can
therefore be labeled as a reconnection routine:
~
\begin{enumerate}
	\item The user signals the intention to resume communications with a USB
	device, on which the disconnection routine has been previously performed.
	\item The HC driver configures the device.
	\item The HC driver enables the capability to schedule new transfers to the
	device.
	\item The operating system is notified that the device is reachable and
	matches it to appropriate drivers, which initiate communications with it.
\end{enumerate}

The specialization of both routines is performed in the same way as other bus
module interactions. All HC drivers hand off their DDF and device callbacks to
the bus module, which then calls them back to perform low-level commands related
to specific devices, endpoints and transfers. This way, the high-level logic
contained by the bus module essentially follows the listed descriptions above
and the version-specific extensions are resolved in the respective HC driver
implementations.


\subsection{Offline and Online DDF Signal}

The HelenOS Device Driver Framework includes two user-initiated signals
relevant to the implementation of this feature.

\begin{description}
	\item[Offline Signal]
		This signal informs a driver attached to a DDF node that its managed
		device may be removed in the near future. The driver is expected to
		immediately cease all user operations on the device and unbind its
		child DDF functions, possibly sending a \textit{Device Remove} signal
		to all their attached drivers in the process.

	\item[Online Signal]
		This signal is a logical counterpart to the previous signal.
		It informs a driver attached to a DDF node that its managed device will
		not be removed in the near future. The driver is expected to expose all
		child DDF functions related to the device, possibly sending a
		\textit{Device Add} signal to all their matched drivers in the process.
\end{description}

These signals can be easily issued by the user from the system shell by means
of the \app{devctl} application. See Listing \ref{lst:devctl-offline-online}
for invocation example.

\begin{listing}[H]
	\begin{bdsh}
		# Prepare the unplug high speed device at address 2.
		devctl offline /hw/pci0/00:04.0/usb2-hs

		# We changed our mind. Bring the device back online.
		devctl online /hw/pci0/00:04.0/usb2-hs
	\end{bdsh}
	\caption[Example usage of \app{devctl} to issue offline and online
	signal.]{Example usage of the \app{devctl} application to issue offline and
	online signal to a USB high speed device at address 2. The host controller
	PCI address is \texttt{00:04.0}.}
	\label{lst:devctl-offline-online}
\end{listing}

It follows that these signals can be used for the implementation of the
explicit device removal at the level of USB host controller drivers. For that
reason, \lib{libusbhost} has been extended to handle appropriate DDF callbacks
for functions corresponding to HC's child devices. Their handling is forwarded
to the bus module, which executes the disconnection or the reconnection routine
for the \textit{offline} and \textit{online} signal respectively. In addition,
the transfer scheduling mechanism of the bus module has been extended to permit
scheduling new transfers only to devices, which are in the online state. The
version-specific part of the implementation is discussed in the next sections.


\subsection{UHCI Specifics}

% TODO @aearsis


\subsection{OHCI Specifics}

% TODO @aearsis


\subsection{EHCI Specifics}

% TODO @aearsis


\subsection{xHCI Specifics}

Since xHCI is the latest HC interface implementation, a lot more is done by the
hardware for the HC driver in comparison with previous versions. The concept of
xHCI command ring leads to very elegant implementation of the required
functionality on the HC driver part.

When the device drivers cease their communications, the HC aborts any leftover
active transfers. For that purpose, the xHCI features an explicit \textit{Stop
Endpoint} command, which instructs the HC to abort all transfers to a specific
device endpoint. This command is issued by the HC driver for all removed device
endpoints, which are active at the moment of the request. Furthermore, device
configuration is dropped along with all remaining endpoints by issuing a
\textit{Configure Endpoint} command with the DC (deconfigure) flag.

Reconnection is quite straightforward and requires only that the HC driver
issues a regular \textit{Configure Endpoint} command in order to transition the
device from the \textit{Addressed} state back to the \textit{Configured} state.


\subsection{Driver Support}

Since the existing USB drivers were quite incomplete, their implementation has
been extended to add support for explicit device removal. This mostly lead to
novel approaches to deallocation of device-related memory structures, which is
often performed in the \fnc{device_remove()} and \fnc{device_gone()} driver
callbacks.

For instance, a number of drivers required that an implementation of
\fnc{device_remove()} is created in the first place. Quite so often, the
existing implementation of \fnc{device_gone()} has provided a good starting
point, given that both functions have to deal with USB device's demise. The
fundamental difference was that while \fnc{device_gone()} merely dealt with
the fallout of unexpectedly unplugged device, \fnc{device_remove()} had the
opportunity to tie up all lose ends prior to the physical detachment of the
device.

This posed a problem, especially in HID and hub drivers, which heavily relied on
polling of interrupt endpoints. Since polling was a synchronous operation from
the device driver's point of view, a polling fibril has been created when the
driver started managing the device. When the device was unplugged, the polling
operation failed with error, waking up and effectively terminating the fibril in
the process. Since no explicit joining mechanism was available, the
implementation of \fnc{device_gone()} merely spinned a limited number of times,
waiting for the polling fibril to die. This mechanism was however unsuitable for
\fnc{device_remove()}, since the fibril would not awaken due to an error caused
by the physical disconnect, which has not happened yet.

This lead to complete refactoring and extension of the USB device polling
mechanism, which is described in detail in Section \ref{polling-refactoring}.
In summation, the new version of the mechanism allows device drivers to join
their polling fibrils and consistently wake them up in order to avoid deadlocks.

The implementation of explicit device removal support for the rest of the
drivers was a trivial extension of their previous functionality.

