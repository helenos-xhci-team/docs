\chapter{Introduction}

\section{About HelenOS}
% TODO:

\section{Drivers in HelenOS}
% TODO: Introduce DDF, terms like device, function, interfaces, add/remove/gone ops, ...

\section{Briefly about USB}
% TODO: Introduce reader to terms like speed, endpoint, transfer type, ...


\section{Existing USB Subsystem}

The support for USB was started by the HelUSB project defended in 2011. In that
time, the USB driver framework was designed, and delivered with few USB
drivers. From the Host Controller side, UHCI and OHCI were supported almost
completely. As for the device drivers, only a generic HID driver was provided
to demonstrate the functionality of the framework. Since the project was
delivered, the USB stack evolved a little and new drivers were implemented.

This document does not aim to replace the original documentation of HelUSB
project. As such, we will focus more on the things that changed since the
documentation was written, and they are pretty much not documented anywhere.
And by all means, information in this chapter is written with regard to the
state before our project was implemented. Thus, great part of the information
given in this section is already obsolete, but it's needed to assess the damage
we're personally responsible for.

The USB framework defines two classes of drivers -- the host controller drivers
and USB function drivers. For the first class, there is a library called
\lib{libusbhost} that aids in providing the unified interface of the host
controller to USB function drivers, and also implements common HC
functionality. There are four HC drivers at the moment:

\begin{itemize}
\item
	\textbf{VHC}, Virtual Host Controller. Implemented in the early phase of
	HelUSB project, served probably as a dummy backend to allow better work
	parallelization and debugging.

\item
	\textbf{UHCI}, Universal Host Controller Interface driver. The earliest
	interface supporting speeds of USB 1.0: Low- and Full-speed devices.
	Important for running HelenOS under QEMU, as it's the interface of the
	default HC QEMU emulates. Apart from isochronous transfers, the driver
	covers all functionality UHCI provides.

\item
	\textbf{OHCI}, Open Host Controller Interface driver. Somewhat complete,
	yet a bit simplified, especially in terms of transfer scheduling. Does not
	care about the polling interval, but schedules all interrupt transfers on
	every frame. Isochronous transfers not supported.

\item
	\textbf{EHCI}, Enhanced Host Controller Interface driver. Mostly a copy of
	OHCI driver, as it uses similar structures. Shall support USB 2 speeds, but
	the support is very limited -- the driver cannot use High-speed hubs to
	communicate with Full- and Low-speed devices, as the support for
	Transaction Translation is completely broken. Also, the bandwidth
	accounting is not implemented for High speed. Neither this controller
	supports isochronous transfers.
\end{itemize}

The HC driver is no longer split in half (as the project documentation states),
but all HC drivers emulate a virtual hub that is driven by a regular
\lib{usbhub} driver. The tree physical topology of USB is kept only inside the
HC driver, and is presented flat to the Device Driver Framework -- all USB
devices are child functions of the HC driver. They communicate with each other
through the DDF driver interface called \texttt{usb\_iface}, which contains all
methods various drivers use.

When the driver enumerates a USB function, it is usually taken care of by the
\lib{usbmid} driver. This driver scans the device descriptor for provided
interfaces, and create children functions for them. These functions are then
driven by class drivers. Notable exception being the \lib{usbhub} driver,
taking care of the device directly, as hubs are not allowed to have multiple
interfaces.

The USB function drivers are well supported by the \lib{libusbdev} library.
This library builds an abstraction layer above \lib{libdrv}, used by other
drivers directly, to better suit needs of USB devices. It does a complete
initialization of the USB device, including initiating a separate IPC
connection to the HC driver directly -- to avoid bouncing all operations in the
\lib{usbmid} driver. For this purpose, the \texttt{usb\_iface} contains two
methods: \fnc{get_my_iface} and \fnc{get_my_interface_handle}. The first one is
answered by the \lib{usbmid} driver with the number of interface driven, or
with a value of $-1$ by the HC driver if the driver serves the whole device.
The \fnc{get_my_interface_handle} call is recursive, until it reaches the HC
driver -- which answers it with devman handle of the USB device function. The
device driver then uses it to initiate connection to the HC driver, the same
way as the usbmid driver do.

Although this is sort-of hacky solution (the devman handle is supposed to be
private), it is currently the only one. Ideally, the drivers would use
a special method to let new connection forward to the HC driver, but for
complicated reasons, this does not work as expected. We discussed this with the
current HelenOS developers, and they confirmed us that the issue is still not
solved.

The \lib{usbhub} driver uses another four methods defined by the
\texttt{usb\_iface}. The interface methods \fnc{reserve_default_address} and
\fnc{release_default_address} ensure synchronization across multiple hubs
(possibly across multiple hub drivers), as the software must ensure that only one
device is listening on the default address at the same time. Then,
\fnc{device_enumerate} and \fnc{device_remove} announce that a device is
connected to (detached from) the hub, to be enumerated (removed) by the HC
driver. The interesting part is that the hub driver has no access to the
created device, as the logical topology presented to the DDF is flat.

All USB device drivers specify the endpoints they expect from the device in
a form of static description, which they pass to the \lib{libusbdev} library
during the driver initialization. Once a new device is added, the library
fetches the device descriptor and matches available endpoints against the
specification provided by the driver. Then the library creates \emph{pipes} --
an abstraction of endpoints based on their properties, not their exact numbers,
which are usually implementation defined. Pipes are then used by the driver
as, well, pipes to push data through and read data from.

The pipe creation process and their usage define the last four methods of which
the \texttt{usb\_iface} is comprised of: \fnc{(un)register_endpoint}, \fnc{read}
and \fnc{write}. The endpoint (un)registration informs the HC driver about
a pipe creation/disposal, and \fnc{read}/\fnc{write} methods are used to
actually transmit packets. Note that the interface is unified regardless of the
transfer type used by the endpoint.

As for the drivers available, there is a solid support for USB HID devices,
implementing keyboards, mice and multimedia keys. Also, a driver for USB Mass
storage exists and somehow works, despite several warnings and errors printed
to the log. Also, a fallback driver is provided to handle any USB device, to
enable the device examination for devices without their own driver, mainly for
debugging purposes.

Not to forget, there are two userspace utilities related to the USB stack. One
of them, \texttt{mkbd}, is not so important, as it is used only to demonstrate
functionality of multimedia keys HID driver. The other one, \texttt{usbinfo}
can be used to list available USB devices:

\begin{bdsh}
/ # usbinfo -l
Bus 37: /hw/pci0/00:04.0/ctl
	Device 61: /hw/pci0/00:04.0/usb1-fs
	Device 65: /hw/pci0/00:04.0/usb2-ls
\end{bdsh}

Other use cases for this utility include descriptor dumping and examination of
device status.

It needs to be said, that whole structure of the USB framework (and also the
DDF framework in general) expects the drivers to behave correctly and does not
implement any countermeasures against malicious behaviour of drivers. For
example, the \texttt{usbinfo} utility connects directly to the HC in the same
way as the device driver does, and fetches the device descriptor. In fact, any
other task can communicate directly with any USB device. Or, any driver can
call the interface methods designed for hubs only -- for example, it can
reserve the default address and never release it. Due to the experimental and
in-development nature of HelenOS, this is not an important problem. Yet, it is
an obstacle to solve before HelenOS will be ready for ``normal'' users, and it
will be a tough one.

Another thing related to the whole USB stack is that support for device
removal is in fact non-existent. At the time of HelUSB project, there was no
support for device removal in the Device Driver Framework, so it's not
surprising that the USB framework inherited this. There are attempts to
terminate interaction and release resources in case of repeated communication
errors.

\section{Goals of the Project}
% TODO


\section{Structure of This Document}
% TODO

