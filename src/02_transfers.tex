\section{Transfers}
\label{sec:transfers}

The xHC uses transfers to abstract USB communication. Every call to
\fnc{usb_write} and \fnc{usb_read} is accomplished by creating a transfer,
setting it up, handing it to the xHC and waiting for transfer completion. This
section describes the code structure and implementation details of the xHCI transfer
subsystem.

\subsection{Executing Transfers}

The xHC is responsible for communicating with USB devices using USB packets.
These USB packets are created from the data given to the xHC through the
transfer rings by the software. The software isn't required to split the
transfer into USB packets, the xHC takes care of that.

Each transfer is formed by a \textit{Transfer Descriptor}, which is a sequence
of TRBs with a chain bit set. This allows the software to schedule a transfer
containing larger data, or data not stored in contiguous memory. Each TRB
points to a single contiguous buffer of data (in physical memory) and specifies
the buffer's size. This buffer must be accessible to the hardware and its
physical address needs to be set to the TRB.

Every endpoint uses a single transfer ring (with the exception of streams)
initialized during the endpoint configuration. This transfer ring is passed to
the xHC by the endpoint context. The software is not allowed to modify the
transfer ring except when it wants to enqueue more transfer descriptors. If the
software requires to modify the ring, it has to first stop the endpoint using
the command interface, and after the modification is done, it has to use the
evaluate context command to communicate the changes to the xHC.

After the software puts a transfer descriptor on the transfer ring, it rings
the doorbell associated with the device and the endpoint. This notifies the xHC
and causes it to schedule the endpoint and process the transfer.

After the transfer descriptor finishes processing, the xHC can optionally
notify the software of the completion by generating a \textit{Transfer Event}
on the event ring with the completion code set to reflect the result. By
default, this only happens if the TRB processing ends with an error.
The software can set the \textit{IOC (Interrupt on Completion)} bit if it needs
to be notified regardless of an error happening.

\subsection{Lifecycle}

Every transfer is represented by a single instance of \struct{xhci_transfer_t}
structure. This structure encapsulates the \struct{usb_transfer_batch_t}
structure, which is used by \lib{libusbhost} to schedule USB batches. Its
lifetime is managed by the library calling the bus operations.

After the library allocates and fills up the \struct{usb_transfer_batch_t}, it
schedules the transfer. This is where the xHCI takes control over the transfer.
The transfer is processed in \fnc{xhci_transfer_schedule}, where it is split
in TRBs, the correct bits are set depending on the transfer type, TRBs are
enqueued and the doorbell is rung.

The fibril responsible for this transfer then waits for the transfer event
signalizing that the transfer has finished. This means that the IOC bit is set
for every transfer descriptor so that the software gets notified about its
completion.

After the transfer event is received, the event loop calls
\fnc{xhci_handle_transfer_event}. There, the TRB which triggered the event is
processed and the transfer, that scheduled it, is found. For IN endpoints, the
received data is copied to prepared buffers. In the end, an error code is set in
the batch and the batch is finalized.

\subsection{Isochronous Transfers}

Due to the nature of isochronous transfers, their implementation needs to avoid
the common steps used by other transfers. This section describes the
requirements and details of the isochronous transfer support.

The xHC has an internal timer and it measures time in \textit{microframes},
where 1 microframe is exactly 125 microseconds. The current time in microframes
can be read anytime from the \texttt{MFINDEX} register. This register has only
14 bits and therefore it wraps every 2.048 seconds. The xHC can notify the
software about the wrap by generating a \textit{MFINDEX Wrap Event} on the
event ring. This feature is optional and must be enabled beforehand.

Every isochronous transfer descriptor must have a set schedule time. This time
is calculated by the software and depends on the isochronous endpoint interval.
Two following transfer descriptors must be always exactly the interval apart.
Because the isochronous endpoints can have up to 4-second interval, we need to
track the \textit{MFINDEX Wrap Events} to correctly determine the scheduling
time.

The xHC gives software strict deadlines when an isochronous transfer descriptor
may be present on the transfer ring for the transfer to execute successfully.
This interval is partially HC specific and it's based on \textit{Isochronous
scheduling interval} (IST). The transfer must be present on the ring at least
\textit{IST} microframes and at most 895 milliseconds before it is to be
executed.

The isochronous endpoint also reserves some bandwidth during its setup and has
only its reserved bandwidth available, therefore the xHC never permits using
more bandwidth. This sets a hard cap on the size of the data transferred in a
single transfer descriptor. Scheduling a transfer descriptor with larger data
causes the xHC to generate an error and refuse to transfer the data.

If the xHC accesses an isochronous transfer ring to retrieve a transfer
descriptor and the transfer ring is empty, the xHC generates a Transfer Event
with the completion code set to \textit{Ring Overrun} for IN endpoints, and
\textit{Ring Underrun} for OUT endpoints. This also removes the endpoint from
the schedule and lets software clean the structure and report the error to the
driver if needed. To reschedule the endpoint, the software needs to ring a
doorbell.

\subsubsection{Implementation}

When we were implementing isochronous transfers, we had to consider all the
requirements described above. We tried to not introduce an API specific for
isochronous transfers to not clutter the interface.

To maintain the \lib{libusbhost} API, we have decided to slightly change the
semantics of calling \fnc{usb_write} and \fnc{usb_read} on isochronous
endpoints. The \struct{xhci_endpoint_t} structure includes a substructure for
fields specific to isochronous endpoints in \struct{xhci_isoch_t}. There is a
small trick present to avoid inflating the \struct{xhci_endpoint_t} for every
type of endpoint. The last field of the structure is a zero-length array of
\struct{xhci_isoch_t}. This doesn't take any space, but the pointer to this
field is always valid and points behind the structure. We then allocate the
structure with \texttt{calloc(1, sizeof(xhci\_endpoint\_t) + (type ==
USB\_TRANSFER\_ISOCHRONOUS) * sizeof(*ep->isoch))}. If the endpoint is
isochronous, the memory allocated is larger to accommodate the
\struct{xhci_isoch_t} and we use the pointer mentioned previously to access it.

The \struct{xhci_isoch_t} also contains a dynamic array of
\struct{xhci_isoch_transfer_t}. These represent an isochronous transfer and
maintain a permanently allocated data buffer. The amount of these depends on
the IST and the endpoint interval as we need to make sure we put the transfer
descriptors on the transfer ring in advance but do not schedule too old data.

\subsubsection{IN endpoints}

For IN endpoints, the semantics of \fnc{usb_read} is changed so that instead
of triggering a read transfer, the call retrieves previously read data and
returns them to the caller. This means that we have to enqueue read transfers
in advance and the caller is expected to withdraw them fast enough. If there is
no prepared data, the call blocks waiting for them instead of returning an
error.

To keep the transfer ring filled with transfer descriptors, we prepare a
transfer descriptor for every transfer buffer when the first read is called and
enqueue as many as possible to the transfer ring before ringing the doorbell
for the first time. This means that not all descriptors may be enqueued because
of the strict deadlines described above. If this happens, we start a timer,
which takes care of putting the transfer descriptors on the ring when the time
is right. After every call to \fnc{usb_read} and withdrawing the received data,
the processed buffer is recycled and scheduled back on the transfer ring.

\subsubsection{OUT endpoints}

For OUT endpoints, the \fnc{usb_write} doesn't ensure that the transfer
finishes. Instead, it only copies the data to the buffer and attempts to put
the buffer on the ring, if possible. If there is no free buffer yet (so all the
buffers are on the ring), the call blocks until at least one transfer is
finished and its buffer is freed.

Unlike with IN endpoints, we need to track, which buffers are filled and which
are empty, and schedule only filled buffers. Other than that, the scheduling is
the same. When scheduling the buffers at the start, we also need to make sure
some of those are already ready on the ring, so we postpone the execution of
the first transfer descriptor by a short delay.

\subsubsection{Error handling}

The transfer errors in isochronous transfers are very often not considered fatal
and should be skipped. Therefore, we decided to not deliver the
error to the driver immediately, instead, the error that occurred is returned with
the next \fnc{usb_read} or \fnc{usb_write} call, in the \texttt{error} field of
\struct{usb_transfer_batch_t}.

On the other hand, the \textit{Ring Overrun} and \textit{Ring Underrun} errors
are considered fatal as there is no simple way to recover from them. For IN
endpoints and therefore for \textit{Ring Overrun}, the error tells us that the
driver either cannot keep up and read the data fast enough, or it has stopped
reading completely. To not preserve old data, we reset the state of our
internal buffers. If the driver attempts to read again, the transfers are
started again as if the endpoint was accessed for the first time.

For OUT endpoints and \textit{Ring Underrun}, this means that the driver cannot
supply data fast enough or it has stopped sending them. We reset the endpoint
again and restart if the driver writes to the endpoint again.

Unfortunately, some host controllers (or QEMU, at least) don't generate
\textit{Ring Overrun} and \textit{Ring Underrun} events. We have therefore
implemented a mechanism to detect the issues these events convey ourselves.
There is a \textit{reset timer} for each isochronous endpoint that is started
every time a buffer is added to the ring, or a transfer event is received. The
timer is set to expire after the endpoint interval passes plus a small
constant. If this timer expires, we know that there was no event for at least
the endpoint interval and therefore the endpoint isn't working properly, so we
consider this as the error and reset the endpoint.

\subsection{Streams}

Streams are a new feature that was introduced with USB 3. They are used to
create another abstraction layer over bulk endpoints in order to give software
an easy way to schedule multiple unrelated transfers in parallel using a single
endpoint. Every bulk endpoint must either work in a mode without streams or
work with stream support. It's not possible for any endpoint to work in both
modes at the same time.

Every device with an endpoint that supports streams declares the support in the
endpoint's superspeed endpoint companion descriptor. The device driver then
configures the endpoint with the stream support to enable streams. A single
bulk endpoint may support up to 65533 streams.

At the host side, every stream has its own transfer ring. When the device
driver communicates through the USB pipe, it can choose a different stream for
every \fnc{usb_write()} and \fnc{usb_read()}.

Streams are scheduled cooperatively over a single bulk pipe. The current stream
selection can be initiated by both host controller and the device. The opposite
side may refuse the selection if for example it doesn't have data buffers
prepared. It is not possible for the system software to select the current
stream manually.

According to the USB3 specification, one possible way to use streams may be
supporting out-of-order data transfers required for mass storage device command
queuing. During the development of xHCI support for HelenOS, we haven't
encountered any USB mass storage device declaring stream support. Therefore,
the current xHCI stream support is completely untested and its API is not
available to the device drivers. The current implementation is described in the
next few paragraphs in case there is a need to write a driver using bulk
streams and debug their support.

\subsubsection{Stream contexts}

As stated above every stream has its own transfer ring which must be made
accessible to the xHC. For regular endpoints, the transfer ring's dequeue
pointer is passed by the endpoint context. With streams, there are many
transfer rings that need to be passed to the xHC. To accomplish that, the
software needs to use stream contexts.

A stream context is a simple structure used by the xHC to store transfer ring
dequeue pointers. There are two possible approaches to setting up stream
contexts. The first approach is called a linear stream array. In this case,
every stream context directly contains a transfer ring dequeue pointer. The
disadvantage of this approach is a need for a large continuous memory buffer as
there may be up to 65536 streams. When selecting a stream, the stream ID set by
system software is used directly as an index to the stream context array.

The second approach uses a hierarchical structure where each primary stream
context may either contain a transfer ring dequeue pointer, or it may point to
a secondary stream context array. In this case, the maximum size of primary
stream context array is 256 stream contexts. The secondary stream context array
can again have up to 256 stream contexts. This allows the software to use
multiple smaller memory buffers instead of one large continuous buffer. The
secondary stream context support is not mandatory and not every host controller
may provide such support. With this approach, the stream ID set by system
software is split into two parts. The first few low order bits (the exact
amount depends on the number of primary streams used) is used as an index to
primary stream context array and the high order bits are used as an index into
secondary stream context array.

In our code, every stream context has an associated instance of
\struct{xhci\_stream\_data\_t}. This structure is used to bookkeep either the
secondary stream context array buffer or the associated transfer ring.

To enable stream support, \lib{libusbhost} should call either
\fnc{xhci\_endpoint\_request\_primary\_streams()} or
\fnc{xhci\_endpoint\_request\_secondary\_streams}. This initializes the stream
contexts and transitions the endpoint to the mode with stream support. The
first function initializes a linear stream context array. The second function
initializes a hierarchical array and its parameter \texttt{sizes} defines the
sizes of individual secondary stream context arrays. These functions assume
that there are no TRBs currently scheduled on the transfer ring of the affected
endpoint. To stop using streams, \lib{libusbhost} should call
\fnc{xhci\_endpoint\_remove\_streams}.

\subsubsection{Handling Transfer Events}

To finish a transfer and report the result to the device driver, we need to
process transfer events which tell us when a transfer finishes. This works well
for any endpoint which doesn't use streams. Nevertheless, if an endpoint uses
streams, we encounter a problem. There is no way to determine the stream, which
generated the transfer event, solely from the dequeued TRB.

Luckily, the xHC gives us a way to pass custom data in the transfer event.
A system software may create a TRB with the TRB Type set to
\textit{Event Data TRB}, which is chained to the previous TRB it is associated
with. This TRB has the IOC bit set instead of the previous TRB and it can bear
any custom 64-bit value. The previous TRB also needs to have
\textit{Evaluate Next TRB} bit set which forces the xHC to always process the
\textit{Event Data TRB} immediately before another stream is selected.

Our software uses \textit{Event Data TRBs} to store the pointer of the transfer 
structure creating this transfer. This allows us to access the transfer
structure in the transfer event handler and determine the selected stream.
